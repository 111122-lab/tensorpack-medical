\documentclass{article}

%----------------------------------------------------------------------
% Notes in the margin on alternating side
\usepackage[textsize=tiny, textwidth=2.5cm]{todonotes}
\usepackage{marginnote}
\newcounter{todonumber}
\newcommand{\todoAmir}[2][]{{%
 \let\marginpar\marginnote%
 \ifodd\value{todonumber}%
   \reversemarginpar%
 \else%
 \fi%
 \todo[#1]{#2}}%
 \stepcounter{todonumber}%
}
%----------------------------------------------------------------------



\title{Plane Detection Using DQN}
\author{Amir Alansary}

\begin{document}
\maketitle

\begin{abstract}
Plane detection using reinforcement learning in medical imaging
\end{abstract}


\section{Training}
\subsection{Preparing ground truth}
Goal is to extract plane parameters and sampled 2d grid image (all done nicely in physical space)
\begin{itemize}
    \item Find the norm form of the ground truth plane equation
    \item Find the origin of the 2d plane by projecting the current 3d origin onto it
    \item Project the corners of the xy-plane containing the 3D origin on the 2d-plane
    \item Sample a 2d grid in the directions of the projected corners around the 2d origin (e.g. grid size = 100 mm)
    \item Store the indeces of the corners to calculate Chebyshev distance (maximum metric) for the reward function
\end{itemize}


\subsection{Play/Train mode}
\begin{itemize}
    \item Intialize 3d origin randomly around the center.
    \item Initialze random plane parameters. \todoAmir{Norm with directional angles between [0,180] and d between}
    \item Extract a new 2d grid around the origin of the new plane
    \item Feed the network and predict new plane params
    \item Repeat steps 3 and 4 till oscillation
\end{itemize}


\end{document}
